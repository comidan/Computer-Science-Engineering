\documentclass[12pt]{article}

\usepackage{indentfirst}

\usepackage[english]{babel}
\usepackage{latexsym}
\usepackage{amsfonts}
\usepackage{amsmath}
\usepackage{amsthm}
\usepackage{amssymb}
\usepackage{amscd}
\usepackage{verbatim}
\usepackage[latin1]{inputenc}
\usepackage{graphicx}
\usepackage{xypic}
\usepackage{euscript}
\usepackage[T1]{fontenc}
\usepackage{fancyhdr}



\def\theequation{\arabic{section}.\arabic{equation}}
\def\thesection{\arabic{section}}



\def\thesubsection{\rm{\arabic{section}.\arabic{subsection}.}}
\newcommand{\dis}{\displaystyle}
\newcommand{\be}{\begin{equation}}
\newcommand{\ee}{\end{equation}}
\newcommand{\bd}{\begin{displaymath}}
\newcommand{\ed}{\end{displaymath}}
\pagestyle{empty}


\def\�{\`{e}}
\def\�{\`{a}}
\def\�{\`{o}}
\def\�{\`{u}}
\def\�{\`{i}}
\def\�{\�{e}}

\begin{document}


\section*{Paradossi}
Provare la scorrettezza dei seguenti ragionamenti.
\begin{enumerate}
  \item Lo zio Joe, lo zio Jim e il loro nipote stanno andando dal barbiere. Nella bottega lavorano tre barbieri: Allen, Brown e Carr. Carr � un'ottimo barbiere, Brown invece � maldestro e Allen, dopo aver avuto una febbre, non ha pi� la mano ferma. Di conseguenza, lo zio Jim vorrebbe farsi radere da Carr e lo zio Joe gli assicura che cos� sar�, visto che � pronto a scommettere mezzo scellino che Carr � certamente nella bottega. Lo zio Jim � scettico e chiede come faccia a esserne cos� sicuro. Lo zio Joe risponde che � in grado di dimostrarlo con la logica, utilizzando soltanto le seguenti due informazioni:
        \begin{enumerate}
          \item La bottega � aperta, quindi almeno uno dei tre barbieri � dentro.
          \item Allen sa che il suo collega Brown � un pessimo barbiere per cui non lascia mai il negozio senza portarlo via con s�.
        \end{enumerate}
      Quindi se Carr fosse fuori, allora se anche Allen fosse fuori, Brown sarebbe dentro, visto che la bottega � aperta e deve essere presente almeno un barbiere; tuttavia, sappiamo che Allen, quando esce, porta con s� Brown. Siccome le asserzioni \lq\lq Se Allen � fuori, allora Brown � dentro\rq\rq\ e \lq\lq Se Allen � fuori, allora Brown � fuori\rq\rq\ sono incompatibili, l'ipotesi che Carr sia fuori conduce a una contraddizione e quindi non pu� essere vera. Ergo, Carr � sempre nella bottega.
      (Lewis Carroll, "A Logical Paradox: by Lewis Carroll", 1894.)
  \item Dice A: \lq\lq se tra le carte di Max c'� un re, allora c'� anche un asso\rq\rq, dice B: \lq\lq se Max non ha un re, allora non ha un asso\rq\rq. Solo uno dei due dice la verit�, ma noi sappiamo che Max ha un re. Ne deduciamo quindi che Max ha un asso. (Illusione cognitiva dei modelli mentali di Johnson-Laird.)
\end{enumerate}


\end{document}
