\documentclass[12pt]{article}

\usepackage{indentfirst}

\usepackage[english]{babel}
\usepackage{latexsym}
\usepackage{amsfonts}
\usepackage{amsmath}
\usepackage{amsthm}
\usepackage{amssymb}
\usepackage{amscd}
\usepackage{verbatim}
\usepackage[latin1]{inputenc}
\usepackage{graphicx}
\usepackage{xypic}
\usepackage{euscript}
\usepackage[T1]{fontenc}
\usepackage{fancyhdr}



\def\theequation{\arabic{section}.\arabic{equation}}
\def\thesection{\arabic{section}}



\def\thesubsection{\rm{\arabic{section}.\arabic{subsection}.}}
\newcommand{\dis}{\displaystyle}
\newcommand{\be}{\begin{equation}}
\newcommand{\ee}{\end{equation}}
\newcommand{\bd}{\begin{displaymath}}
\newcommand{\ed}{\end{displaymath}}
\pagestyle{empty}


\def\�{\`{e}}
\def\�{\`{a}}
\def\�{\`{o}}
\def\�{\`{u}}
\def\�{\`{i}}
\def\�{\�{e}}

\begin{document}


\section*{Monoide}

\noindent Utilizzando un opportuno linguaggio del primo ordine si scrivano gli assiomi di monoide e si traduca la seguente frase con una formula chiusa:\\
\lq\lq Se un elemento \� invertibile a sinistra, allora \� cancellabile a sinistra.\rq\rq\\
Si ricordi che un elemento $x$ si dice cancellabile a sinistra se ogni volta che vale l'uguaglianza $xy=xz$, allora si ha $y=z$.)\\ \ \\
Si scriva nella sintassi di SPASS un programma che permetta di verificare se l'enunciato proposto \� vero nella teoria dei monoidi.\\ \ \\
\begin{itemize}
    \item Costanti: $b$ (unit\�)
    \item Funzioni: $P(x,y)$ (prodotto)
\end{itemize}
\[
\forall x(\exists y(P(y,x)=b)\Rightarrow \forall u\forall v ((P(x,u)=P(x,v))\Rightarrow (u=v)))))
\]









\end{document}
