\documentclass[12pt]{article}

\usepackage{indentfirst}

\usepackage[english]{babel}
\usepackage{latexsym}
\usepackage{amsfonts}
\usepackage{amsmath}
\usepackage{amsthm}
\usepackage{amssymb}
\usepackage{amscd}
\usepackage{verbatim}
\usepackage[latin1]{inputenc}
\usepackage{graphicx}
\usepackage{xypic}
\usepackage{euscript}
\usepackage[T1]{fontenc}
\usepackage{fancyhdr}



\def\theequation{\arabic{section}.\arabic{equation}}
\def\thesection{\arabic{section}}



\def\thesubsection{\rm{\arabic{section}.\arabic{subsection}.}}
\newcommand{\dis}{\displaystyle}
\newcommand{\be}{\begin{equation}}
\newcommand{\ee}{\end{equation}}
\newcommand{\bd}{\begin{displaymath}}
\newcommand{\ed}{\end{displaymath}}
\pagestyle{empty}


\def\�{\`{e}}
\def\�{\`{a}}
\def\�{\`{o}}
\def\�{\`{u}}
\def\�{\`{i}}
\def\�{\�{e}}

\begin{document}


\section*{Paralogismi}
Provare la scorrettezza dei seguenti ragionamenti.
\begin{enumerate}
  \item Tutti gli uomini sono animali, alcuni animali sono carnivori, dunque alcuni uomini sono carnivori.\\
      (Aristotele, Elenchi sofistici)
  \item Se qualcuno � a Megara, non � ad Atene; ma v'� un uomo a Megara; dunque non v'� un uomo ad Atene.\\  (Diogene Laerzio, Vite, VII, 186) 
  \item Se qualcosa � rubato, non � guadagnato n� si � avuto in modo appropriato, dunque se non � guadagnato n� si � avuto in modo appropriato, � rubato.\\   (Pietro Ispano, Trattato di logica, VII, 158)
  \item Tutti i quadrangoli sono figure geometriche,\\ 
  ma nessun triangolo � un quadrangolo;\\ 
  dunque, alcune figure geometriche non sono triangoli.\\ \ \\
  Ma qui il vero svolgimento del ragionamento � questo:\\ \ \\ 
Regola: Tutti i quadrangoli non sono triangoli.\\
Caso: Alcune figure geometriche sono quadrangoli.\\
Risultato: Alcune figure geometriche non sono triangoli.\\ 
(Charles Sanders Peirce, Deduzione, induzione e ipotesi)

\end{enumerate}


\end{document}
